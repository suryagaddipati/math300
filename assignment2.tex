\documentclass[12pt]{article}
\usepackage{amsmath, amssymb,amscd}
\usepackage{setspace}
\usepackage{enumerate}
\usepackage{amsmath}
\usepackage{ amssymb }
\pagestyle{empty}
\setlength{\parindent}{0in}
\oddsidemargin 0in
\textwidth 6.5in
\topmargin 0in
\textheight 9in
\doublespace



\begin{document} 

Surya Gaddipati\\
\today\\
MATH 300 - Monday

\begin{center}
Derivative
\end{center}
\paragraph{}
 When you finish a workout on a treadmill you get to see your final speed at which you were running. But you were running at different speeds thoughout the workout and you are interested in finding out what your speed was at a particular instant not your overall speed.    
 \paragraph{}
 Speed can be calculated as change in distance divided by change in time. \\
  $ speed = \dfrac{\vartriangle distance}{\vartriangle time}$ \\
  Let's assume $f(x)$ is a function which gives the distance travelled at time $x$.
  To calculate your speed at a particular moment, lets pick two instants surronding the instant where you would like to calcuate your speed.  
  Let $a$ be the instant number one and $b$ be instant number two surrounding our target instant. $b$ can be described as $a + (b -a)$  or $a + \vartriangle t$

  \begin{equation} 
    velocity at target instant = \dfrac { f(b) - f(a) } { b -a }
  \end{equation}
  
  But we would like to know our speed at that particular instant not between two arbitary points surrounded by that point in time. We would like $b-a$ to be $0$. We could use limits to calcualte the function as $b-a$ approaches zero. 
  \begin{equation}
    \lim_{(b -a) \to 0} \dfrac { f(b) - f(a) } { b -a } 
  \end{equation}
 $ \dfrac{dx}{dy} $ is the notation used to describe the 'instantaneous' change of a function. 
 Derivate answers the question 'At what rate value of something was changing?'. In example above speed gives the rate at which distance is changing.


  
\end{document}
