\documentclass[12pt]{article}
\usepackage{amsmath, amssymb,amscd}
\usepackage{setspace}
\usepackage{enumerate}
\usepackage{amsmath}
\usepackage{ amssymb }
\usepackage{amsthm}
\pagestyle{empty}
\setlength{\parindent}{0in}
\oddsidemargin 0in
\textwidth 6.5in
\topmargin 0in
\textheight 9in
\doublespace

\begin{document} 

Surya Gaddipati\\
\today\\
MATH 300 - Monday
\newtheorem*{KL}{Theorem }

\begin{KL} 
  Let $f$ be a function from the set of real numbers to itself. Prove that if $f$ is differentiable at a real number $c$, then $f$ is continuous at $c$.
\end{KL}
\begin{proof}
  Given that $f$ is differentiable at $c$. 
  \begin{equation}
    \lim_{h \to 0} \dfrac { f(c+h) - f(c) } {h} 
    \end{equation}
exists. Let's call that $f'(c)$. 
In order to prove that $f$ is continous function at $c$ we need to prove that $f'(c)=f(c)$.
 \paragraph{}
  \begin{equation}
    \lim_{h \to 0}  f(c+h) - f(c)  =  \lim_{h \to 0} \dfrac { f(c+h) - f(c) } {h}  . h 
    \end{equation}
  \begin{equation}
    \lim_{h \to 0}  f(c+h) - f(c)  =  f'(c) . h 
    \end{equation}
  \begin{equation}
    \lim_{h \to 0}  f(c+h) - f(c) =   f'(c) . 0 = 0 
    \end{equation}

  \begin{equation}
    \lim_{h \to 0}  f(c+h) =  f(c)  
    \end{equation}
which means that $f$ is continous at $c$.
\end{proof}
\end{document}
