\documentclass[12pt]{article}
\usepackage{amsmath, amssymb,amscd}
\usepackage{setspace}
\usepackage{enumerate}
\usepackage{amsmath}
\usepackage{ amssymb }
\usepackage{amsthm}
\pagestyle{empty}
\setlength{\parindent}{0in}
\oddsidemargin 0in
\textwidth 6.5in
\topmargin 0in
\textheight 9in
\doublespace

\begin{document} 

Surya Gaddipati\\
\today\\
MATH 300 - Monday
\begin{center}
Mathematical Paradox
\end{center}
\paragraph{}
A paradox is a statment which seems absurd or self-contradictory that when investigated or explained may prove to be well founded or true. A mathematical paradox is a mathematical propostion which intutively sounds absurd but turns out to be true upon proof.
\paragraph{}
 Lets examine \textbf{Simpson's paradox} as a example of a mathematical paradox.  Its a pradox in which a trend that appears in different groups of data disappears when these groups are combined, and the reverse trend appears for the aggregate data. 
 \paragraph{}
 What we mean by that is that the intuitive conculusion that we come to by examining results of individual groups might be opposite of what aggregate result says.
 For example, lets say we are trying to decide between team A and team B as to which team is a better team  based on their previous history of wins and losses. And we are give the following data of their past performances in the last two seasons.
 \\
 \begin{tabular}{l*{2}{c}r}
   Team              & A & B \\
   \hline
    Season 1 & 63/90 (70\%) & 8/10 (80\%)  \\
   Season 2  & 4/10(40\%) & 45/90 (50\%)   \\
 \end{tabular}
 \\
 Just by looking at the table above we intutively come to a conculsion that team B is a better team. But if we look the aggregate results, team A has won 67/100 ( 67\%) while team B won 53/100( 53\%) games. This contradicts our intuition that team A is a better team.
\end{document}

