\documentclass[12pt]{article}
\usepackage{amsmath, amssymb,amscd}
\usepackage{setspace}
\usepackage{enumerate}
\pagestyle{empty}
\doublespace
\usepackage{amsthm}



\begin{document} 

Surya Gaddipati\\
September 1, 2013\\
MATH 300 

\newtheorem*{KL}{Theorem }

\begin{KL} 
Given a positive integer $n$, suppose $S$ is a subset of \{$1$, $2$,...,$2n$\} with $|S| = n + 1$. Prove that there are distinct $a$,$b$ in $S$ such that $a$ divides $b$.
\end{KL}
\begin{proof}
  Let $k$ be the smallest number of set $S$.\\
  $k$ cannot be be greater than $n$ for set $S$ to contain $n+1$ elements.
  \begin{equation}
    k \leq n  
  \end{equation}
  \begin{equation}
    k \leq n + 1
  \end{equation}
  Adding $k$ to both sides of the equation, we get 
  \begin{equation}\label{eq:solve}
    2k \leq k+ n + 1
  \end{equation}

  Equation~\ref{eq:solve} implies that smallest member of the set $k$ and $2k$ are contained in $S$ since $k+n+1$ is the largest member of $S$. Which proves the fact that $S$ contains two distinct elements $a$ and $b$ such that $a|b$ since $k|2k$. 

\end{proof}

\end{document}
end{document}
