\documentclass[12pt]{article}
\usepackage{amsmath, amssymb,amscd}
\usepackage{xypic}
\usepackage{setspace}
\usepackage{enumerate}
\pagestyle{empty}
\setlength{\parindent}{0in}
\oddsidemargin 0in
\textwidth 6.25in
\topmargin 0in
\textheight 9in
\doublespace
\usepackage{amsthm}



\begin{document} 

Surya Gaddipati\\
September 1, 2013\\
MATH 300 

\newtheorem*{KL}{Theorem }

\begin{KL} 
  $\sqrt 2$ is irrational 
\end{KL}
\begin{proof}
  Lets assume that $\sqrt 2$ is rational, meaning  it can be expressed as 
  \begin{equation}
    \sqrt{2} = \frac{a}{b}
  \end{equation}

  where $a$ and $b$ are real numbers.\\
  This implies,
  \begin{equation}
    2 = \frac{a^2}{b^2}
  \end{equation}
  \begin{equation}\label{eq:solve}
    2b^2 = a^2
  \end{equation}

  Thus $a^2$ is an even number which means $a$ is an even number, since squares of even numbers are even numbers.
  Since $a$ is an even number there exists a number $c$ such that 
  \begin{equation}
    a= 2c
  \end{equation}

  Substituting, $c$ in ~\ref{eq:solve} we get

  \begin{equation}
    2b^2 = (2c)^2
  \end{equation}
  \begin{equation}
    2b^2 = 4c^2
  \end{equation}
  \begin{equation}\label{eq:biseven}
    b^2 = 2c^2
  \end{equation}
  Equation ~\ref{eq:biseven} implies that $b$ is an even number too. This means that $a$ and $b$ have a common factor of $2$, but $a$ and $b$ shouldn't have any common factors.\\
  We have a contradiction. Hence $\sqrt 2$ is irrational.

\end{proof}

\end{document}
